\input Fiee_style


\starttext

\TitlePage
  { {\tfd \sc CARTOGRAFÍA}
\blank[3cm]
   
      {\bf L. Mauricio Reyna B.}\\
      {lizardo.reyna@utm.edu.ec}\\
      \blank[big]
      {\bf Roger Delgado}\\
      {roger.delgado@utm.edu.ec}

 

  \blank[big]

  \vfill
    
    
  {\tfx Abril 2023}

  }


\Topic{Preámbulo}  
  



\startitemize
\item Direcciones...?
\item Norte, Sur, Este, Oeste
\item Indique la dirección del Campus Lodana...
\item Dibuje un croquis ...
\stopitemize
\blank[2cm]
{\bf Esta clase te ayudará a entender sobre mapas}


\Topic{Objetivos}

\startitemize
\item ... expresar sus conocimientos sobre mapas cartográficos
\item ... comprender la importancia de los mapas
\item ... utilizar nuevo vocabulario para interpretar cartografía
  \item ... analizar su entorno geográfico
\stopitemize

\Topic{Contexto}

\bTABLE[frame=off]
\bTR
\bTD[width=8cm] En la actualidad, los mapas son esenciales para el día
día.
\blank[big]
Las app's nos brindan muchas facilidades para interpretar mapas de
localización.

\blank[big]

... aunque no todos los mapas son únicamente para localización\eTD  
\bTD \externalfigure [figs/map01][width=9cm] \eTD  
\eTR
\eTABLE

{\it La cartografía es la ciencia (el arte) de hacer mapas ...}

\Nopic{¿Porqué la cartografía es importante?}

... {\it una imagen vale más que ...}

      {\bf --:: Información geográfica de forma digital o en papel}\\
      {\bf --:: Posición, ubicación, distancias, superficies, tamaños ...}\\
      {\bf --:: Cantidades, cualidades, características...}\\
      {\bf --:: Temporalidad...}

      \blank[3cm]
      {\it ... el mapa es el segundo lenguaje de la geografía...! N. Baranski}
\blank[2cm]

     \goto{https://booth.lse.ac.uk/}[url(https://booth.lse.ac.uk/)]

      \bTABLE[frame=off]
      \bTR\bTD {\externalfigure [figs/map03][width=.7\textwidth]}
      \eTD\eTR
      \bTR
      \bTD {\externalfigure [figs/map02][width=.7\textwidth]} \eTD  
      \eTR
      \eTABLE


      \Nopic{Sitios web}

       \goto{website 01}[url(https://images-teaching.is.ed.ac.uk/luna/servlet/view/all?sort=work_creator_name,work_title,work_display_date,work_subject_place)]

       \goto{https://www.davidrumsey.com/}[url(https://www.davidrumsey.com/)]

       \goto{http://www.ub.edu/gehc/es/}[url(http://www.ub.edu/gehc/es/)]

       \goto{https://islandskort.is/}[url(https://islandskort.is/)]

       \goto{https://www.visionofbritain.org.uk/maps/}[url(https://www.visionofbritain.org.uk/maps/)]

       \goto{https://www.raremaps.com/}[url(https://www.raremaps.com/)]

       \Nopic{¿Qué hay en un mapa?}

             {\bf --:: Aspectos culturales}\\
             -- Lagunas, caminos, ríos, edificaciones, etc.

             {\bf --:: Información numérica}\\
             -- población, salarios, edades, peso, etc.

             {\bf --:: Aspectos físicos}\\
             -- vegetación, cobertura, geología, geomorfología,
             suelos, clima, etc

             \blank[3cm]

                   {\bfd Datos / información}

                   \page
                   ~
                       \blank[5cm]
                   \startalignment[middle]
                     {\bfc¿Cómo se representan esos datos?}

                     \goto{https://solarsystem.nasa.gov/resources/2393/earth-3d-model/}[url(https://solarsystem.nasa.gov/resources/2393/earth-3d-model/)]
                   \stopalignment
                   \page

                   {\externalfigure
                     [figs/world3d][width=.9\textwidth]}

                   \Topic{Proyecciones}

                   {\externalfigure
                     [figs/projections][width=.8\textwidth]}

                   \Topic{Escalas}

                         {\bf --::  Representación numérica}\\
                         -- 1:50000\\

                         {\bf --:: Verbal}\\
                         -- 1cm = 500m

                         {\bf --:: Gráfica}\\

                         \blank[]

                               {\externalfigure [figs/scale][width=8cm]}


           \Topic{Tipos}

                 {\bf --:: Temáticos}\\

                 Representan una o pocas variables específicas;
                 geología, cobertura, población, etc.

                 {\bf --:: Topográficos}\\

                 Representa con precisión características naturales,
                 físicas, culturales, etc, en un espacio geográfico.

                 \hairline

                 \startitemize[packed]
                 \item Puntos
                 \item Líneas
                   \item Polígonos
                 \stopitemize

                
                               
% \startFLOWchart[example]
% \startFLOWcell 
%   \name {fia}
%   \location {2,1}
%   \text {FIA}
%   \connection [bt] {pregrado}
%   \connection [bt] {posgrado}
  
% \stopFLOWcell
% \startFLOWcell 
%   \name {pregrado}
%   \location {1,2}
%   \text {PREGRADO}
%   \connection [ll] {agricola}
%   \connection [ll] {tec}
% \stopFLOWcell
% \startFLOWcell
%   \name {posgrado}
%   \location{3,2}
%   \text {POSGRADO}
%   \connection [rr] {mia}
%   \connection [rr] {mec}
%    \connection [rr] {agro}
% \stopFLOWcell
% \startFLOWcell
%   \name {agricola}
%   \location{1,3}
%   \text {Ingeniería Agrícola}
%   \connection [rl] {lab}
%   \connection [rl] {centro}
%    \connection [rl] {salas}
% \stopFLOWcell
% \startFLOWcell
%   \name {tec}
%   \location{1,4}
%   \text {Tecnologías Geoespaciales}
% \stopFLOWcell
% \startFLOWcell
%   \name {mia}
%   \location{3,3}
%   \text {Ingeniería Agrícola}

% \stopFLOWcell
% \startFLOWcell
%   \name {mec}
%   \location{3,4}
%   \text {Mecanización Agrícola}
% \stopFLOWcell
% \startFLOWcell
%   \name {agro}
%   \location{3,5}
%   \text {Agroecología}
% \stopFLOWcell
% \startFLOWcell
%   \name {lab}
%   \location{2,2}
%   \text {Lab. Suelos/Agua}
% \stopFLOWcell
% \startFLOWcell
%   \name {centro}
%   \location{2,3}
%   \text {Centro de Mecanización}
% \stopFLOWcell
% \startFLOWcell
%   \name {salas}
%   \location{2,4}
%   \text {Datos, 3D y SIG}
% \stopFLOWcell
% \startFLOWcell
%   \name {resi}
%   \location{1,5}
%   \text {Residencia}
% \stopFLOWcell
% \stopFLOWchart
% \FLOWchart[example]





\stoptext
